\section{Ungleichungen für Wkeiten}
$\mathds{P}[\Omega] = 1$\\
$\mathds{P}[A_1\cup A_2 \cup ...] = \mathds{P}[A_1] + \mathds{P}[A_2]+ ...$\\
falls $A_i \cap A_j = \emptyset \: \forall i+j$\medskip\\
\textbf{7.}:\\
$\forall  A \subset B \subset \Omega \Rightarrow \mathds{P}[A] \leq \mathds{P}[B]$\\
\textbf{Bew.}: $B=A\cup(B\backslash A)$ (disjunkt)\\
$\mathds{P}[B]=\mathds{P}[A]+\underbrace{\mathds{P}[B\backslash A]}_{\geq 0} \geq \mathds{P}[A] \qed$\medskip\\
\textbf{8.}:\\
$\forall A, B \subset \Omega : \mathds{P}[A\cup B] \leq \mathds{P}[A] + \mathds{P}[B]$\medskip\\
\textbf{Allgemeiner}: $\forall A_1,...,A_n \subset \Omega : \mathds{P}[A_1\cup ...\cup A_n] \leq \mathds{P} [A_1]+...+\mathds{P}[A_n]$\medskip\\
\textbf{9. } Noch allgemeiner:\\
$\forall A_1,A_2,... \subset \Omega : \mathds{P}[\bigcup_{k=1}^\infty A_k] \leq \Sigma_{k=1}^\infty \mathds{P}[A_k]$\\
\textbf{Bew.}:\\
$B_1 = A_1$\\
$B_2 = A_2\backslash A_1$\\
$B_3 = A_3\backslash(A_1 \cup A_2)$\\
...\smallskip\\
Dann gilt:\\
$\underbrace{\bigcup_{k=1}^\infty A_k}_\text{Nicht disj.} = \underbrace{\bigcup_{k=1}^\infty B_k}_\text{disj.!!!}$\medskip\\
$\mathds{P}[\bigcup_{k=1}^\infty A_k] = \mathds{P}[\bigcup_{k=1}^\infty B_k] \overset{\text{Axiom}}{=}\Sigma_{k=1}^\infty \mathds{P}[B_k] \underset{B_k \subset A_k}{\leq} \Sigma_{k=1}^\infty \mathds{P}[A_k] \qed$
\chapter{Bedingte Wahrscheinlichkeiten und Unabhängigkeiten}
\textbf{Bsp.}: 2 faire Würfel \hspace{1cm} $\Omega = \{1,...,6\}^2 \quad \#\Omega=36$\\
A= 'Erster Würfel zeit eine 6''\hspace{1cm} B=''Augensumme = 10''\smallskip\\
Jemand teilt uns mit: B ist eingetreten.\smallskip\\
$A=\{(6,1),(6,2),...,(6,6)\} \quad \#A =6 \quad \mathds{P}[A] = \frac{6}{36}= \frac{1}{6}$
$B=\{\underbrace{(6,4)}_\text{A tritt ein},(5,5),(4,6)\}$\medskip\\
$\mathds{P}[A|B]= \frac{1}{3}$\medskip\\
\textbf{Definition}:\\
Seien $A,B \subset \Omega$. Bedinge Wkeit von A gegeben B ist: $$\mathds{P}[A|B] = \dfrac{\mathds{P}[A \cap B]}{\mathds{P}[B]}$$
Annahme: $\mathds{P}[B] \neq 0$\medskip\\
\textbf{Bsp.}:\\ Fairer Würfel wird 10x geworfen.\\
Uns wird mitgeteilt, dass mindestens eine 6 gewürfelt wurde.\smallskip\\
Bedingte Wkeit, dass der erste Wurf eine 6 war = ?\medskip\\
\textbf{Lösung}:\\
$\Omega = \{(a_1,...,a_{10}):a_i \in \{1,..,6\}\} \quad \#\Omega = \overbrace{6*6*...*6}^{10} = 6^{10}$\\
B = ''mind. eine 6 gewürfelt'' \hspace{1cm} $B^C =$ ''keine 6 gewürfelt'' =\\
$\{(a_1,...,a_10):a_i \in \{1,2,...,5\}\} \quad \#B^C = 5^{10}$\smallskip\\
$\#B = 6^{10}-5^{10}$\\
$\mathds{P}[B] = \dfrac{6^{10}-5^{10}}{6^{10}}$\smallskip\\
A = '' der erste Wurf ist eine 6''\\
$A = \{(6,a_2,...,a_10):a_i \in \{1,...,6\}\}$\\
$\#A = \underbrace{1*6*6*...*g}_9 = 6^9	$\smallskip\\
$\mathds{P}[A] = \dfrac{6^9}{6^{10}}= \dfrac{1}{6}$\medskip\\
$A\cap B = A$\\
$\mathds{P}[A\cap B] = \mathds{P}[A]=\frac{1}{6}$\smallskip\\
$\mathds{P}[A  \vert B ] = \dfrac{\mathds{P}[A \cap B]}{\mathds{P}[B]} = \dfrac{6^9}{6^{10}-5^{10}} = 0,19$\medskip\\
$\mathds{P}[A\vert B] = 0,19$\\
$\mathds{P}[A] = 0,16$\medskip\\
\textbf{Alternativ}: $\mathds{P}[A\vert B] = \dfrac{\#(A\cap B)}{\#B}$
\section{Eigenschaften der bedingten Wkeiten}
\begin{enumerate}
	\item $\mathds{P}[A \vert B] = \dfrac{\mathds{P}[A \cap B]}{\mathds{P}[B]} \leq 1 $ (und $\geq 0$)
	\item $\mathds{P}[\Omega\vert B] = \dfrac{\mathds{P}[\Omega \cap B]}{\mathds{P}[B]}= \dfrac{\mathds{P}[B]}{\mathds{P}[B]} = 1$
	\\ $\mathds{P}[\emptyset \vert B] = 0$
	\item Falls $A_1,A_2,...$ disj. sind, gilt:
	$$\mathds{P}\left[\left(\bigcup_{k=1}^\infty A_k\right)\vert B\right] = \Sigma_{k=1}^\infty \overset{\text{Def.}}{=} \dfrac{\mathds{P}\left[\left(\bigcup_{k=1}^\infty A_k\right)\cap B\right]}{\mathds{P}[B]}$$
	$$= \dfrac{\mathds{P}\left[\bigcup_{k=1}^\infty(A_k \cap B)\right]}{\mathds{P}[B]}$$
	$$\overset{A_\cap B ,A_2 \cap B,... \text{disj}}{=} \dfrac{\Sigma_{k=1}^\infty \mathds{P}[A_k \cap B]}{\mathds{P}[B]} = \Sigma_{k=1}^\infty \dfrac{\mathds{P}[A_k \cap B]}{\mathds{P}[B]} = \Sigma_{k=1}^\infty \mathds{P}[A_k \vert B] \qed$$
	\item $\mathds{P}[A^C\vert B] = 1 - \mathds{P}[A\vert B]$
	\item Multiplikationsregel:\\$\mathds{P}[A_1\cap A_2] = \mathds{P}[A_1]*\mathds{P}[A_2\vert A_1]$\smallskip\\
	$\mathds{P}[A_1\cap A_2\cap A_3] = \mathds{P}[A_1] * \mathds{P}[A_2 \vert A_1]* \mathds{P}[A_3 \vert (A_1 \cap A_2)]$\smallskip\\
	\textbf{Sterbewahrscheinlichkeiten $q_0,q_1,...$}\\
	$q_n = $ Wahrscheinlichkeit, als eine n-jährige Person, dass Alter n+1 nicht zu erreichen.\smallskip\\
	$q_0 = 0,0046$ \hspace{2cm} Wkeit, dass eine Person $\geq$ 50 alt wird\\
	$q_1 = 0,0004$\\
	$q_2 = 0,0002$\smallskip\\
	\textbf{Lösung}: Def.: $A_n$=''Person hat das Alter von n Jahren erreicht''\\$\mathds{P}[A_{50}] = ?$\medskip\\
	Gegeben sind $q_n = \mathds{P}\left[A_{n+1}^C \vert A_n\right]\quad 1-q_n = \mathds{P}[A_{n+1}\vert A_n]$\\
	$\mathds{P}[A_{50}] = \mathds{P}[A_1]*\mathds{P}[A_2 \vert A_1]*\mathds{P}[A_3\vert (A_1 \cap A_2)]*\mathds{P}[A_4 \vert \underbrace{(A_1 \cap A_2 \cap A_3)}_{A_3}] * ... * \mathds{P}[A_{50}\vert \underbrace{(A_1\cap...\cap A_{49})}_{A_{49}}] $\\ $=(1-q_0)*(1-q_1)*...*(1-q_{49})$\medskip\\
	$\mathds{P}[\text{Person lebt genau 50 Jahre}] =(1-q_0)*(1-q_1)*...*(1-q_{49}) * q_{50}$\qed
\end{enumerate}

