\chapter{Kombinatorik}
\textbf{Beispiel}: Geburtstagsproblem \hspace{1.5cm} K = 200 Personen\\
A = ''mindestens 2 Personen haben am gleichen Tag Geburtstag''\\
$\mathds{P}[A] = ?$\\\\
\textbf{Modell}: n = 365 Tage im Jahr\\\\
\textbf{Ausgänge}: Liste der Länge K besteht aus Zahlen zwischen 1 und n\medskip\\
$\Omega = \{1,...,n\}^K=\{(a_1,...,a_k):a_i \in \{1,...,365\}\forall i = 1,...,k\}$\smallskip\\
$a_i = $ Geburtstag der i-ten Person.\\
\#$\Omega =\underbrace{n*n*...*n}_K = n^K = 365^{200}$\\
$\mathds{P}[A] = \frac{\#A}{\#\Omega}$\smallskip\\
Gegenereignis $A^C$ = ''alle Geburtstage sind (paarweise) verschieden''\\\\
$\#A^C = \underbrace{n}_{\text{Mögl. für Person 1}}*\underbrace{(n-1)}_{\text{Mögl. für Person 2}}*\underbrace{(n-2)}_\text{Mögl. für 3. Person} * ... * \underbrace{(n-K+1)}_\text{Mögl. für Person k} = (n)_K$ mit K$ \leq  $n\medskip\\
$\mathds{P}[A] = \frac{\#A}{\#\Omega} = \frac{\#\Omega-\#A^C}{\#\Omega}=1-\frac{\#A^C}{\#\Omega}=1-\frac{n(n-1)...(n-K+1)}{n^K}$\medskip\\
\textbf{Beispiel}:\\ K = 23 Personen: $\mathds{P}[A]=$0.507\\K = 200: $\mathds{P}[A]=0.999999...8 \approx 1$\newpage
\section{Urnenmodelle}
Urne mit Bällen 1,...,n. Es wird k mal jeweils ein Ball zufällig gezogen. [*GRAFIK URNE]\\
\textbf{Möglichkeiten}:
\begin{enumerate}
	\item[a)] Mit/ohne Zurücklegen
	\item[b)] Nummern der Bälle mit/ohne Berücksichtigung der Reihenfolge
\end{enumerate}
Insgesamt 4 Modelle:
\begin{enumerate}
	\item \textbf{Mit Zurücklegen und mit Berücksichtigung der Reihenfolge}\\
	Ausgänge dieses Experiments sind Listen der Länge K bestehend aus Zahlen zwischen 1 und n.\\
	$\Omega=\{(a_1,...,a_k):a_i \in \{1,...,n\} \forall i = 1,..,K\}$\\
	$a_i$ = die Nummer des i-ten Ball\smallskip\\
	\textbf{Beachte}
	\begin{itemize}
		\item 	Es ist möglich, dass $a_i = a_j$ (mit Zurücklegen)
		\item 	$(5,3,7,...) \neq (3,5,7,...)$
	\end{itemize}
	\textbf{Beispiel}:
	\begin{itemize}
		\item Geburtstage von Personen. Bälle = Tage. \\Jede Person zieht einen Ball zufällig.
		\item k-maliges Würfeln. 6 Bälle = 6 Seiten des Würfels\\
	\end{itemize}
	$\#\Omega = \underbrace{n * n * ... * n}_K = n^K$
	\item \textbf{Ohne Zurücklegen und mit Berücksichtigung der Reihenfolge}\\
	$\Omega=\{(a_1,...a_k):a_i \in \{1,...,n\},\underbrace{a_i \neq a_j\forall i\neq j}_\text{Paarweise verschiedene Elemente} \}$\smallskip\\
	\#$\Omega = n* (n-1)*(n-2)*...*(n-K+1)$\smallskip\\
	\textbf{Beachte}
	\begin{itemize}
		\item (1,3,\textbf{2},4,5,\textbf{2},...) $\notin \Omega$
		\item (5,3,7,...) $\neq$ (3,5,7,...)
	\end{itemize}
	\textbf{Bemerkungen}: 
	\begin{itemize}
		\item Falls k $>$ n: \#$\Omega = 0$
		\item Für k = n:\\
		\#$\Omega = n*(n-1)*...*1 = 1*2*3*...*n = n!$\\Ausgänge sind Permutationen:\\n= 3:(1,2,3),(1,3,2),(2,1,3),(2,3,1)
	\end{itemize}
	\item \textbf{Ohne Zurücklegen und ohne Berücksichtigung der Reihenfolge}\\
	Ausgänge: Listen der Länge K aus verschiedenen Elementen.\\
	Allerdings wird die Reihenfolge nicht berücksichtigt, d.h. \{2,5,3\} = \{5,3,2\}\smallskip\\
	Ausgänge sind K-elementige Teilmengen von \{1,...,n\}.\\
	$\Omega = \{A:A\subset\{1,...,n\},|A| = k\}$ oder \\
	$\Omega = \{(a_1,...,a_k):a_i \in \{1,...,n\},\underbrace{a_1<a_2<...<a_k}_{\substack{\text{Reihenfolge wird}\\\text{ durch sortieren gelöscht}}} \}$\medskip\\
	\#$\Omega = \frac{n*(n-1)*...*(n*K+1)}{K!}$ \hspace{0.5cm}K Objekte aus n Objekten auswählen\smallskip\\
	$ = \binom{n}{k}$\medskip\\
	\textbf{Beispiel}: Lotto, 49 Kugeln. 6 Kugeln werden ohne Zurücklegen gezogen.\\ 
	Wir tippen auf eine Kombination aus 6 Nummern\smallskip\\
	A = ''Man hat alle 6 geraten'' $\mathds{P}[A] = \frac{1}{\binom{49}{6}}$\medskip\\
	\textbf{1. Lösung}:\\
	$\Omega = $ Menge aller 6-elementigen Teilmengen von \{1,...,49\}. \\
	Die Kugeln werden mit einem Griff gezogen.\medskip\\
	\#$\Omega = \binom{49}{6} \qquad \#A = \{1 \text{[die Kombination, auf die wir tippen]}\}$\\
	$\mathds{P}[A] = \frac{1}{\binom{49}{6}} \approx 7,15*10^{-8}$\medskip\\
	\textbf{2. Lösung}:\\
	Kugeln werden einzeln gezogen, Nummern werden notiert:\\
	$\Omega = \{(a_1,...,a_6):a_i \in \{1,...,49\},a_i \neq a_i \forall i \neq j\}$\smallskip\\
	\#$\Omega = 49*48*47*...*(49-6+1) \qquad \#A = 6! $\medskip\\
	Wir tippen auf \{1,...,6\}. wir gewinnen bei allen Permutationen von 1,...,6.\\
	$\mathds{P}[A] = \frac{\#A}{\#\Omega}=\frac{6!}{49*48*...*44} = \frac{1}{\binom{49}{6}}$\medskip\\
	\textbf{Beispiel}:\\ Wie hoch ist die Chance, dass 2 mal in Folge die gleichen Zahlen gezogen werden?\\
	Bis zum Zeitpunkt gab es K = 3016 Ziehungen.\\
	Insgesamt gibt es $\binom{49}{6}$ Gewinnreihen.\\
	A = ''Bei mindestens 2 Ziehungen wurde die gleiche Reihe gezogen''\\$\approx$ Geburtstagsproblem\\
	$A^C = $ ''Alle Ziehungen ergeben verschiedene Reihen''\\
	$\mathds{P}[A] = 1 - \frac{n(n-1)...(n-K+1)}{n^K}$ = 0,278\\
	$\Rightarrow$ \textbf{Ferni-Dirar-Statistik}
	\newpage
	\item \textbf{Mit zurücklegen und ohne Reihenfolge}\\
	K Vögel setzen sich auf n Bäume
	\begin{itemize}
		\item 	Mehrfachbesetzungen möglich
		\item 	Vögel identisch
	\end{itemize}
	\textbf{Wieviele Besetzungen sind möglich?}\medskip\\
	\textbf{Lösung}:\\
	Insgesamt $\underbrace{K}_\text{Vögel}+\underbrace{n-1}_\text{''Trennwände''}$ Symbole, davon K Kreuze\medskip\\
	\#$\Omega = \binom{K+n-1}{K} = \binom{K+n-1}{n-1} $\medskip\\
	$\binom{a}{b} = \binom{a}{a-b}$\smallskip\\
	$\Rightarrow$ Bose-Einstein-Statistik (für diese Vorlesung unwichtig)
\end{enumerate}

