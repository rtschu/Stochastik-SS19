\chapter{Varianz und Kovarianz}
\textbf{Definition}: Sei X eine ZV. Varianz von X: \hspace{1cm} Var X=$\mathds{E}[(X-\mathds{E}X)^2]$\medskip\\
\textbf{Beispiel}: 
Sei X uniform verteilt auf \{-a,a\}\smallskip\\
$\mathds{E}X=\dfrac{1}{2}*a+\dfrac{1}{2}(-a)=0$\smallskip\\
Var X = $\mathds{E}[(x-\mathds{E}X)^2]=\mathds{E}[X^2]=a^2$\medskip\\
\textbf{Definition}: Standardabweichung von X: $\sigma(X)=\sqrt{\text{Var X}}$\medskip\\
Wenn X konstant ist, so ist die Varianz = 0
\section{Satz 14.1} Var X = $\mathds{E}[X^2]-(\mathds{E}X)^2$\medskip\\
\textbf{Kor.} $\mathds{E}[X^2] \geq (\mathds{E}X)^2$\medskip\\
\textbf{Beweis}: Var X = $\mathds{E}[(X-\mathds{E}X)^2] = \mathds{E}[\underbrace{X^2}_\text{A}-\underbrace{2X*\mathds{E}X}_\text{B}+\underbrace{(\mathds{E}X)^2}_\text{C}]0$\smallskip\\
$=\mathds{E}[X^2]-\mathds{E}[2X\mathds{E}X]+\mathds{E}[(\mathds{E}X)^2]$\smallskip\\
$=\mathds{E}[X^2]-2 \mathds{E}X*\mathds{E}[X]+(\mathds{E}X)^2$ da C konstant\smallskip\\
$=\mathds{E}[X^2]-(\mathds{E}X)^2$
\section{Satz 14.2}Sei X ZV, a,b, $\in \mathds{R}$\\Dann gilt
\begin{enumerate}
	\item Var(aX+b)=Var(aX)=$a^2$VarX
	\item $\sigma$(aX+b)=$\sigma$(aX)=$\vert$a$\vert\sigma$(X) \hspace{1cm} $\sqrt{a^2}=\vert a\vert$
\end{enumerate}
\textbf{Beweis 1}:\\
Var(aX+b) $\underset{\text{def}}{=}\mathds{E}\left[\left(aX+b-\mathds{E}[aX+b]\right)^2\right]$\smallskip\\
$=\mathds{E}\left[\left(aX+b-a\mathds{EX-b}\right)^2\right]=\mathds{E}[a^2(x-\mathds{E}X)^2]$\smallskip\\
$=a^2\mathds{E}[(X-\mathds{E}X)^2]=a^2 \text{Var X}$\medskip\\
\textbf{Beweis 2}:\\
$\sigma(aX+b) = \sqrt{\text{Var}(aX+b)} \underset{1}{=}\sqrt{a^2\text{Var X}}= \underbrace{\sqrt{a^2}}_{|a|}\underbrace{\sqrt{Var X}}_{\sigma(X)}$\medskip\\
\textbf{Beispiel}: Sei x uniform verteilt auf \{1,2,\dots,n\}\\
D.h. $\mathds{P}[x=k] = \dfrac{1}{n}\quad \forall k=1,2,\dots,n$\\Var X= ?\medskip\\
\textbf{Lösung}: Var X=$\mathds{E}[x^2]-(\mathds{E}[x])^2$\smallskip\\
$\mathds{E}x=\dfrac{n+1}{2}$\medskip\\
$\mathds{E}[x^2] = \sum_{k=1}^nk^2\mathds{P}[x=k]= \dfrac{1^2+2^2+\dots+n^2}{n}$\smallskip\\
$=\dfrac{n(n+1)(2n+1)}{6n}=\dfrac{(n+1)(2n+1)}{6}$\smallskip\\
Var x = $\dfrac{(n+1)(2n+1)}{6}- (\dfrac{n+1}{2})^2 = \dots = \dfrac{n^2-1}{12}$\medskip\\
\textbf{Exkurs über die Teleskopmethode}
$1^2+2^2+\dots+n^2 = ? \quad \sum_{k=1}^nk^2 = ?$\smallskip\\
Methode: Finde Funktion F mit $k^2 = F(k)-F(k-1)$\\
$1^2+2^2+\dots+n^2=F(1)-F(0)+F(2)-F(1)+F(3)-F(2)+\dots+F(n)-F(n-1)$\smallskip\\
$=F(n)-F(0)$\medskip\\
$k^2 = F(k)-F(k-1) \leftarrow \text{ Ziel}$\smallskip\\
$F(k) = \dfrac{k^3}{3}+ak^2+bk+c \quad a,b,c \text{ unbekannt}$\smallskip\\
$F(k)-F(k-1) = \dfrac{k^3}{3}+ak^3+bk+c-\dfrac{(k-1)^3}{3}-a(k-1)^2-b(k-1)-c$\smallskip\\
$=(k^2-k+\dfrac{1}{3}+a(2k-1)+b)$\smallskip\\
$=k^2+k*(2a-1)+(b+\dfrac{1}{3}-a)$ soll gleich $k^2$ sein.\smallskip\\
$2a-1 = 0 \Rightarrow a\ = \dfrac{1}{2}$\smallskip\\
$b+\dfrac{1}{3}-a=0 \Rightarrow b = \dfrac{1}{6}$\smallskip\\
$F(k) = \dfrac{k^3}{3} = \dfrac{k^2}{2}+\dfrac{1}{6}k+c$\medskip\\
$1^2+2^2+\dots+n^2=F(n)-F(0)=\dfrac{n^3}{3}+\dfrac{n^2}{2}+\dfrac{n}{6}+c-c = \dfrac{n(n+1)(2n+1)}{6}$\medskip\\
\textbf{Beispiel}: Sei X $\sim$Poi($\lambda$)\\
Dann gilt $\mathds{E}$X=Var X = $\lambda$\medskip\\
\textbf{Beweis}: $\mathds{E}X=\lambda $ ist bekannt: \hspace{1cm} $\mathds{E}[X^2]=?$\smallskip\\
$\mathds{E}[X^2]=\sum_{k=0}^\infty k^2\mathds{P}[x=k]=\sum_{k=0}^\infty k^2e^{-\lambda} \dfrac{\lambda^k}{k!}=? $\medskip\\
$\mathds{E}[X(X-1)] = \sum_{k=0}^\infty k(k-1)\mathds{P}[x=k] = \sum_{k=0}^\infty k(k-1)e^{-\lambda} \dfrac{\lambda^k}{k!}$\medskip\\
$=\sum_{k=2}^\infty k(k-1)e^{-\lambda}\dfrac{\lambda^k}{1*2*\dots*(k-1)k}=e^{-\lambda}* \sum_{k=2}^\infty\dfrac{\lambda^k}{(k-1)!}$\smallskip\\
\begin{math}
=e^{-\lambda}*\lambda^2*\sum_{k=2}^\infty \dfrac{\lambda^{k-2}}{(k-2)!}\smallskip\\
=e^{-\lambda} \lambda^2 * + \E x = \lambda ^2+\lambda\smallskip\\
\var\:x = \E[x^2]-(\E x)^2 = \lambda^2+\lambda-\lambda^2=\lambda \qed \sum_{l=0}^{\infty}\dfrac{\lambda^l}{l!}=e^{-\lambda}\lambda^2e^\lambda = \lambda^2 \medskip\\
\E[x^2]=\E[\lambda (x-1)]
\end{math}