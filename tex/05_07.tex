\chapter{Harmonische Reihe}
\section{Harmonische Reihe}
\newcommand{\harmonisch}{\frac{1}{2}+\frac{1}{3}+\frac{1}{4}+\dots}
\textbf{Definition}: Harmonische Zahlen $H_n = 1 + \frac{1}{2}+\frac{1}{3}+\frac{1}{4}+\dots+\frac{1}{n}$\smallskip\\
Harmonische Reihe: $\frac{1}{2}+\frac{1}{3}+\frac{1}{4}+\dots = \underset{n\rightarrow\infty}{\text{lim} }H_n$\medskip\\
\subsection{Satz 10.1}
$\harmonisch = + \infty \qquad H_n = \harmonisch$\medskip\\
d.h. $\forall$ A (egal wie groß) $\exists$ n = n(A) : $H_n \geq A$\smallskip\\
\textbf{Beweis}:\\
\img{0.7}{harm}\\
$H_2 \geq 1 + \frac{1}{2}, H_4 \geq 1+\frac{1}{2}+\frac{1}{2}, H_8 \geq 1+\frac{1}{2}+\frac{1}{2}+\frac{1}{2}$\\
$H_{2^l}\geq 1+\frac{l}{2}$\smallskip\\
$A=7*10^6 \quad 1+\frac{l}{2} \geq A\quad l \geq 2A-2$\\
$H_{2^{1+10^6-2}} \geq 7 * 10^6$
\subsection{Satz 10.2, Euler, 1734}
$\underbrace{\harmonisch+\frac{1}{n}}_{H_n\rightarrow\infty}-\underbrace{\text{log }n}_{\rightarrow\infty} $ konvergiert für n $\rightarrow$ $\infty$ gegen eine Konstante.\medskip\\
\textbf{Bemerkung}: Die Konstante $ \gamma $ := $\underset{n\rightarrow\infty}{\text{0}}(H_n-\text{log } n) = 0,57721\dots $ heißt\\
Euler-Mascheroni-Konstante\smallskip\\
\textbf{Beweis}: $\int_1^n \dfrac{1}{x} dx$ = (log x) $  \big|_1^n$ 1 n = log n - $\underbrace{\text{log 1}}_{0}$ = log n\smallskip\\
\img{0.7}{log}\\
$H_{n-1}  - \text{log } n = $Fläche $S_n$\\
$S_1<S_2<S_3<\dots \qquad$\\$ \text{Außerdem: }S_n \leq 1\: \forall n \in \mathds{N}$
\img{0.3}{1}\\
D.h. $\exists$ lim $S_n$ exisitert und ist $\leq$1 und $\geq$ 0\medskip\\
$H_n - \text{log }n = \underbrace{H_{n-1}-\text{log }n}_{\text{Konv. für n}\rightarrow\infty} + \underbrace{\frac{1}{n}}_0$, also konvertgiert $H_n - \text{log }n$ gegen einem Limes, der zwischen 0 und 1 liegt. \qed\medskip\\
$\harmonisch+\dfrac{1}{n} = \text{log }n + \gamma + \underbrace{\varepsilon_n}_\text{Fehler der Appr.}\text{, wobei }\varepsilon_n \rightarrow 0 \text{ für } n\rightarrow\infty$
\section{Alternierende harm.Reihe}
\newcommand{\altharm}{1-\frac{1}{2}+\frac{1}{3} - \frac{1}{4}+\dots}
Alternierende harm. Reihe: $\altharm = log 2$\medskip\\
\textbf{Beweis}:\\
$S_{2n} = \altharm - \frac{1}{2n}$\smallskip\\
$S_{2n+1}= \altharm -  \frac{1}{2n} + \frac{1}{2n+1}$\medskip\\
$S_{2n} = \altharm - \frac{1}{2n} = H_{2n}-2(\frac{1}{2}+\frac{1}{4}+\dots+\frac{1}{2n})$\smallskip\\
$=H_{2n} - (1+\frac{1}{2}+\frac{1}{3}+\dots+\frac{1}{n})=H_{2n}-H_n = $\smallskip\\
$\underset{\text{Kor.}}{=} (\text{log }2n+ \gamma + \varepsilon_{2n}) - (\text{log }n + \gamma + \varepsilon_n)$\smallskip\\
$=\text{log }2n - \text{log }n + \varepsilon_{2n}-\varepsilon_n$\smallskip\\
$=\text{log }2 + \underbrace{\varepsilon_{2n}}_0-\underbrace{\varepsilon_n}_0\quad$
$\underset{n\rightarrow\infty}{\rightarrow} log 2$\medskip\\

\textbf{Aufgabe}: Schnecke und Auto\\
$1+\dfrac{1}{2}+\dfrac{1}{3}+\dots = \infty$\medskip\\
\imgHere{schneckeUndAuto}\\
\textbf{Zeige}: Schnecke überholt das Auto in endlicher Zeit\\
$\altharm$ = log 2\smallskip\\
$\underbrace{1+\frac{1}{3}-\frac{1}{2}}+\underbrace{\frac{1}{5}+\frac{1}{7}-\frac{1}{4}}+\underbrace{\frac{1}{9}+\frac{1}{11}-\frac{1}{6}}\dots = \frac{3}{2}\text{ log 2}$\medskip\\
Reihe $\sum_{n=1}^{\infty}$ heißt absolut konvergent, wenn $\sum_{n=1}^\infty |a_n|<\infty$\\
\textbf{Beispiel}:$\altharm $ ist konvergent aber nicht absolut konvergent.\medskip\\
\textbf{Satz}: In einer absolut konvergenten Reihe kann man die Terme umordnen, ohne dass sich die Summe ändert.\\
\textbf{Riemann'scher Umordnungssatz}: $\forall a \in \mathbb{R}$ gibt es eine Umordnung der Reihe $\altharm$, die gegen a konvergiert.\medskip\\
\textbf{Beweis}: Sei $a\geq 0$\\
Ungerade Terme: 1, $\frac{1}{3}, \frac{1}{5},\dots \quad 1 + \frac{1}{3}+\frac{1}{5}\dots = \infty$\medskip\\
Gerade Terme: $\frac{1}{2},\frac{1}{4},\dots \quad \frac{1}{2}+\frac{1}{4}+\dots = \infty$\medskip\\
Nehme ungerade Terme bis die Summe $\geq$ a wird: 
$$1+\dfrac{1}{3}+\dfrac{1}{5}+\dots+\frac{1}{n_1}\geq a$$
Ziehe gerade Terme abm bis die Summe $\leq$ a wird
$$1+\dfrac{1}{3}+\dfrac{1}{5}+\dots+\frac{1}{n_1}-\dfrac{1}{2}-\dfrac{1}{4}-\dots-\frac{1}{m_1}\leq a$$
Addiere ungerade Terme bis die Summe wieder $\geq$ a wird
$$1+\dfrac{1}{3}+\dfrac{1}{5}+\dots+\frac{1}{n_1}-\dfrac{1}{2}-\dfrac{1}{4}-\dots-\frac{1}{m_1}+\dfrac{1}{n_1+2}+\dfrac{1}{n_1+4}+\dots +\dfrac{1}{n_2}\geq a$$usw. \qed\medskip\\
\textbf{Bemerkung} Über EWert von ZV mit abzählbar $\infty$-vielen Werten\smallskip\\
Betrachte ZV X: \hspace{1cm} Werte: \begin{tabular}{c|c|c|c}
	$x_1$&$x_2$&$x_3$&\dots\\\hline
	$p_1$&$p_2$&$p_3$&\dots
\end{tabular}\medskip\\
$\mathds{E}X=\sum_{i=1}^\infty x_ip_i ?$ \\
\textbf{Problem:} Keine ''kanonische'' Reihenfolge der Werte $x_i$ Wenn $\sum_{i=1}^\infty$ nicht \textbf{absolut} konv. $\Rightarrow$kann sich das Ergebnis durch Umordnung ändern\medskip\\
\textbf{Definition}: Sei X ZV wie oben Def.
$$S_+ = \sum_{i:x_i \geq 0}x_ip_i,\quad S_ = \sum_{i:x_i < 0}|x_i|p_i$$
\textbf{Fälle}:$
\begin{cases}
	 \mathds{E}X\overset{\text{def}}{=}\sum_{i=1}^\infty x_ip_i = S_+-S_-,  &\text{falls }S_+ < \infty, s_- < \infty\\
	 \mathds{E}X\overset{\text{def}}{=}+\infty,&\text{falls } S_+=\infty, s_- < \infty\\
	\mathds{E}X \overset{\text{def}}{=}-\infty,&\text{falls } S_+ < \infty, S_- = \infty\\
\mathds{E}X \text{nicht definiert }, & \text{falls } S_+ = \infty, S_-=\infty
\end{cases}$
